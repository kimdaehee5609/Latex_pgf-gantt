



















\documentclass[12pt,a3paper,landscape,oneside]{book}

		% --------------------------------- 페이지 스타일 지정
		\usepackage{geometry}
		\geometry{top=20mm}
		\geometry{bottom=20mm}
		\geometry{left=10mm}
		\geometry{right=10mm}
		\geometry{headheight=0.4in}
		\geometry{headsep=0.1in}
		\geometry{footskip=0.3in}
	%	\geometry{showframe}
%	
%		\newgeometry{ 	top=8em, bottom=8em,
%						left=8em, right=8em, 
%						headheight=2em, headsep=2em}
%

		%	===================================================================
		%	package
		%	===================================================================
%			\usepackage[hangul]{kotex}				% 한글 사용
			\usepackage{kotex}						% 한글 사용
			\usepackage[unicode]{hyperref}			% 한굴 하이퍼링크 사용
			\usepackage{amssymb,amsfonts,amsmath}	% 수학 수식 사용


%			\usepackage{tikz}%
%			\usetikzlibrary{arrows,positioning,shapes}

			\usepackage{pgfgantt}
			\usepackage{pdflscape}

		

		% --------------------------------- 페이지 스타일 지정

		\usepackage[Bjornstrup]{fncychap}

		\usepackage{fancyhdr}
		\pagestyle{fancy}
		\fancyhead{} % clear all fields
		\fancyhead[LO]{\tiny \leftmark}
		\fancyhead[RE]{\tiny \leftmark}
		\fancyfoot{} % clear all fields
		\fancyfoot[LE,RO]{\large \thepage}
		%\fancyfoot[CO,CE]{\empty}
		\renewcommand{\headrulewidth}{1.0pt}
		\renewcommand{\footrulewidth}{0.4pt}
	
	
	
		% --------------------------------- 	section 스타일 지정
	
		\usepackage{titlesec}
		
		\titleformat*{\section}		{\large\bfseries}
		\titleformat*{\subsection}		{\normalsize\bfseries}
		\titleformat*{\subsubsection}	{\normalsize\bfseries}
		\titleformat*{\paragraph}		{\normalsize\bfseries}
		\titleformat*{\subparagraph}	{\normalsize\bfseries}
	
		\renewcommand{\thesection}		{\arabic{section}.}
		\renewcommand{\thesubsection}	{\thesection\arabic{subsection}.}
		\renewcommand{\thesubsubsection}{\thesubsection\arabic{subsubsection}}
		
		\titlespacing*{\section} 		{0pt}{1.0em}{1.0em}
		\titlespacing*{\subsection}	  	{0ex}{1.0em}{1.0em}
		\titlespacing*{\subsubsection}	{0ex}{1.0em}{1.0em}
		\titlespacing*{\paragraph}		{0ex}{1.0em}{1.0em}
		\titlespacing*{\subparagraph}	{0ex}{1.0em}{1.0em}
	
	%	\titlespacing*{\section} 		{0pt}{0.0\baselineskip}{0.0\baselineskip}
	%	\titlespacing*{\subsection}	  	{0ex}{0.0\baselineskip}{0.0\baselineskip}
	%	\titlespacing*{\subsubsection}	{6ex}{0.0\baselineskip}{0.0\baselineskip}
	%	\titlespacing*{\paragraph}		{6pt}{0.0\baselineskip}{0.0\baselineskip}
	

		% --------------------------------- recommend		섹션별 페이지 상단 여백
		\newcommand{\SectionMargin}{\newpage  \null \vskip 2cm}
		\newcommand{\SubSectionMargin}{\newpage  \null \vskip 2cm}
		\newcommand{\SubSubSectionMargin}{\newpage  \null \vskip 2cm}


	
		% ----------------------------- 장의 목차
		\usepackage{minitoc}
		\setcounter{minitocdepth}{1}    	% Show until subsubsections in minitoc
		\setlength{\mtcindent}{12pt} 		% default 24pt
	
	
		% --------------------------------- 	문서 기본 사항 설정
		\setcounter{secnumdepth}{3} 		% 문단 번호 깊이
		\setcounter{tocdepth}{3} 			% 문단 번호 깊이
		\setlength{\parindent}{0cm} 		% 문서 들여 쓰기를 하지 않는다.
		
		
		% --------------------------------- 	줄간격 설정
%		\doublespace
%		\onehalfspace
%		\singlespace
		
		
% 	============================================================================== List global setting
%		\setlist{itemsep=1.0em}
	
% 	============================================================================== enumi setting

%		\renewcommand{\labelenumi}{\arabic{enumi}.} 
%		\renewcommand{\labelenumii}{\arabic{enumi}.\arabic{enumii}}
%		\renewcommand{\labelenumii}{(\arabic{enumii})}
%		\renewcommand{\labelenumiii}{\arabic{enumiii})}






		% --------------------------------- recommend  글자 색깔지정 명령
		\newcommand{\red}{\color{red}}			% 글자 색깔 지정
		\newcommand{\blue}{\color{blue}}		% 글자 색깔 지정
		\newcommand{\black}{\color{black}}		% 글자 색깔 지정
		\newcommand{\superscript}[1]{${}^{#1}$}

	
	

% ------------------------------------------------------------------------------
% Begin document (Content goes below)
% ------------------------------------------------------------------------------
	\begin{document}
	
			\dominitoc
			
	\begin{titlepage}
		\vspace*{2cm}
		\centering 
		\Huge {Gantt chart}\\
		\vspace{1cm}
		\Huge {사용 설명서 }\\
		\vfill
		\Large {2015년 7월 9일 목요일 }\\
		\vfill
		\Large {서영엔지니어링 }\\
		\Large {김대희}\\
		\vspace{1cm}
	\end{titlepage}


%			\tableofcontents
%			\listoffigures
%			\listoftables
%
			


% ================================================= chapter 	====================
	\newpage
	\chapter{chapter name}



	% ------------------------------------------ section ------------ 
	\newpage
	\section{일일 야간 작업 스케쥴 분석}

			% -------------------------------------
			\begin{ganttchart}[	hgrid, 
								vgrid,
%								inline
								x unit=6mm,
								y unit chart=0.6cm,
								y unit title=0.8cm,
								time slot format=isodate,
%								compress calendar,
							]{2015-6-12}{2015-7-31}
%			\gantttitlecalendar{year,month,week=3}\\
			\gantttitlecalendar{year,month,week}\\
			% -------------------------------------
%			\ganttbar{안전 헨스 설치}	{2015-6-10}{2015-6-20} \\
%			\ganttbar{안전 헨스 설치}	{2015-6-1}{2015-11-30} \\

			\ganttbar{안전 헨스 설치}	{2015-6-2}{2015-7-2} \\
			\ganttbar{안전 헨스 설치}	{2015-6-5}{2015-7-30} \\
			\ganttbar{안전 헨스 설치}	{2015-6-5}{2015-7-30} \\
			\ganttbar{안전 헨스 설치}	{2015-6-5}{2015-7-30} \\
			\ganttbar{안전 헨스 설치}	{2015-6-5}{2015-7-30} \\
			\ganttbar{안전 헨스 설치}	{2015-6-5}{2015-7-30} \\
			\ganttbar{안전 헨스 설치}	{2015-6-5}{2015-7-30} \\
			\ganttbar{안전 헨스 설치}	{2015-6-5}{2015-7-30} \\
			\ganttbar{안전 헨스 설치}	{2015-6-5}{2015-7-30} \\
			\ganttbar{안전 헨스 설치}	{2015-6-5}{2015-7-30} \\
			\ganttbar{안전 헨스 설치}	{2015-6-5}{2015-7-30} \\
			\ganttbar{안전 헨스 설치}	{2015-6-5}{2015-7-30} \\
			\ganttbar{안전 헨스 설치}	{2015-6-5}{2015-7-30} \\
			\ganttbar{안전 헨스 설치}	{2015-6-5}{2015-7-30} \\
			\ganttbar{안전 헨스 설치}	{2015-6-5}{2015-7-30} \\
			\ganttbar{안전 헨스 설치}	{2015-6-5}{2015-7-30} \\
			\ganttbar{안전 헨스 설치}	{2015-6-5}{2015-7-30} \\
			\ganttbar{안전 헨스 설치}	{2015-6-5}{2015-7-30} \\

			\end{ganttchart}






	% ------------------------------------------ section ------------ 
	\newpage
	\section{일일 야간 작업 스케쥴 분석}

			% -------------------------------------
			\begin{ganttchart}[	hgrid, 
								vgrid,
%								inline
								x unit=0.8cm,
								y unit chart=0.6cm,
								y unit title=0.8cm,
							]{1}{12}
			% -------------------------------------
			\gantttitle{야간 작업 일정}{12} \\
			\gantttitlelist{6,7,8,9,10,11,12,01,02,03,04,05}{1} \\
			% -------------------------------------
			\ganttbar{안전 헨스 설치}	{ 2}{ 2} \\
			\ganttgroup{굴착}{2}{7} \\
			\ganttbar{컷팅}			{ 2}{ 3} \\
			\ganttbar{포장 깨기}		{ 3}{ 4} \\
			\ganttbar{초기 굴착}		{ 5}{ 5} \\
			\ganttbar{흙막이 설치}		{ 6}{ 6} \\
			\ganttbar{최종 굴착}		{ 7}{ 7} \\

			\ganttgroup{관부설}		{ 3}{ 8} \\
			\ganttbar{관 부설}			{ 3}{ 8} \\

			\ganttgroup{되메우기 및 포장}	{ 9}{12} \\
			\ganttbar{모래 되메우기관}		{ 9}{ 9} \\
			\ganttbar{토사 되메우기관}		{ 9}{10} \\
			\ganttbar{보조기층}			{11}{11} \\
			\ganttbar{택 코팅}				{13}{13} \\
			\ganttbar{기층}				{13}{13} \\
			\ganttbar{안전 헨스 철거}		{12}{12} \\

			\end{ganttchart}

			\begin{itemize}
			\item 컷팅 일 작업량 ? ( 주간에 작업 가능 한지 ? ), 컷팅 깊이는 ?
			\item 포장 깨기를 전구간 다 할것인지 ?
			\item 굴착 및 관부설 한사이클 당 진행 거리 ?
			\item 장비 구성은 ?
			\item 택코팅 양생은 ?

			\end{itemize}




			    \begin{ganttchart}{1}{12}
			    \gantttitle{2012}{12} \\
			    \gantttitlelist{1,...,12}{1} \\
			    \ganttgroup{Group 1}{1}{7} \\
			    \ganttbar{Task 1}{1}{2} \\
			    \ganttlinkedbar{Task 2}{3}{7} \ganttnewline
			    \ganttmilestone{Milestone}{7} \ganttnewline
			    \ganttbar{Final Task}{8}{12}
			    \ganttlink{elem2}{elem3}
			    \ganttlink{elem3}{elem4}
			    \end{ganttchart}




			\newpage
			% -------------------------------------
			\begin{ganttchart}[	hgrid, 
								vgrid,
%								inline
								x unit=0.5cm,
								y unit chart=1.0cm,
								y unit title=2.0cm,
							]{1}{23}
			% -------------------------------------
			\gantttitle{야간 작업 일정}{23} \\
			\gantttitle{2015.06}{3}
			\gantttitle{2015.07}{4}
			\gantttitle{2015.08}{4}
			\gantttitle{2015.09}{4}
			\gantttitle{2015.10}{4}
			\gantttitle{2015.11}{4}


 \\

%			\gantttitlelist{6,7,8,9,10,11,12,01,02,03,04,05}{1} \\
%                                     7       8          9          10          11      
			\gantttitlelist{ 	2,3,4,1,2,3,4,1,2, 3, 4, 1, 2, 3, 4, 1, 2, 3, 4, 1, 2, 3, 4}{1} \\
%							1 2 3 4 5 6 7 8 9 10 11 12 13 14 15 16 17 18 19 20 21 22 23  

			% -------------------------------------
			\ganttbar{하천횡단구간}	{11}{19} 
			\ganttbar{}			{20}{23} \\

			\ganttbar{1구간}		{ 8}{10} 
			\ganttbar{}			{20}{23} \\


			\ganttbar{2구간}		{ 9}{10} 
			\ganttbar{}			{20}{23} \\


			\ganttbar{3구간}		{ 4}{ 7} 
			\ganttbar{}			{20}{23} \\


			\ganttbar{4구간}		{ 4}{ 8} 
			\ganttbar{}			{20}{23} \\


			\ganttbar{5구간}		{11}{19} 
			\ganttbar{}			{20}{23} \\

			\ganttgroup{굴착}{2}{7} \\
			\ganttgroup{관부설}		{ 3}{ 8} \\
			\ganttbar{관 부설}			{ 3}{ 8} \\


			\end{ganttchart}




	% ------------------------------------------ section ------------ 
	\newpage
	\begin{landscape}
			% -------------------------------------
			\begin{ganttchart}[	hgrid, 
								vgrid,
								x unit=0.6cm,
								y unit chart=1.2cm,
								y unit title=1.0cm,
%								inline
								newline shortcut=false,
								bar label node/.append style={align=left},
							]{1}{23}
			% -------------------------------------
			\gantttitle{에정 공정표}{23} \ganttnewline
			\gantttitle{2015.06}{3}
			\gantttitle{2015.07}{4}
			\gantttitle{2015.08}{4}
			\gantttitle{2015.09}{4}
			\gantttitle{2015.10}{4}
			\gantttitle{2015.11}{4}\ganttnewline

%			\gantttitlelist{6,7,8,9,10,11,12,01,02,03,04,05}{1} \\
%                                     7       8          9          10          11      
			\gantttitlelist{ 	2,3,4,1,2,3,4,1,2, 3, 4, 1, 2, 3, 4, 1, 2, 3, 4, 1, 2, 3, 4}{1} \ganttnewline
%							1 2 3 4 5 6 7 8 9 10 11 12 13 14 15 16 17 18 19 20 21 22 23  

%			\ganttalignnewline 
			% -------------------------------------

			\ganttbar[name=b00, progress=none]	{\textbf{하천횡단구간} \\ \tiny NO. 0+00.00  \(\sim\) NO. 3+7.38  }
											{11}{19} \ganttnewline

			\ganttbar[name=b01, progress=none]	{\textbf{1구간} \\ \tiny NO. 0+00.00 \(\sim\)  NO. 3+7.38}	
											{ 8}{10} \ganttnewline


			\ganttbar[name=b02, progress=none]	{\textbf{2구간} \\ \tiny NO.11+00.00 \(\sim\)  NO.15+28.52}	
											{ 9}{10} \ganttnewline


			\ganttbar[name=b03, progress=none]	{\textbf{3구간} \\ \tiny NO.15+28.52 \(\sim\)  NO.26+12.55}	
											{ 4}{ 7} \ganttnewline


			\ganttbar[name=b04, progress=none]	{\textbf{4구간} \\ \tiny NO.26+12.55 \(\sim\) NO.38+3.43}	
											{ 4}{ 8} \ganttnewline


			\ganttbar[name=b05, progress=none]	{\textbf{5구간} \\ \tiny NO.38+ 3.43 \(\sim\) NO.50+ 0.00}		
											{11}{19} \ganttnewline

			\ganttbar[name=b06, progress=none]	{기계,전기}			
											{20}{23} 

		    \ganttlink[link type=f-s]{b01}{b00} 
		    \ganttlink[link type=f-s]{b03}{b01} 
		    \ganttlink[link type=f-s]{b04}{b02} 
		    \ganttlink[link type=f-s]{b02}{b05} 

			\end{ganttchart}


	\end{landscape}





























% ------------------------------------------------------------------------------
% End document
% ------------------------------------------------------------------------------
\end{document}


